\documentclass[letter,12pt]{article}
\usepackage[letterpaper,right=1.25in,left=1.25in,top=1in,bottom=1in]{geometry}
\usepackage{setspace}

\usepackage[utf8]{inputenc}   % allows input of special characters from keyboard (input encoding)
\usepackage[T1]{fontenc}      % what fonts to use when printing characters       (output encoding)
\usepackage{amsmath}          % facilitates writing math formulas and improves the typographical quality of their output
\setlength{\parindent}{0pt}   % drop paragraph indentations

\begin{document}

(A) da trato igualitario a los votantes: si dos cambian sus votos, la decisión colectiva queda igual.

(N) da trato igualitario a las altenativas: revierte todos los votos, cambiando cada 1 por $-1$ y cada $-1$ por 1, la decisión colectiva queda igual.

(E) aporta decisividad: si hubiera empate y algunos abstencionistas votaran 1, ceteris paribus, basta para romperlo en favor de 1.

Sin estas propiedades, la regla sesgaría la decisión colectiva en pro de algunos votantes o alternativas, o sería gratuitamente indecisiva.\\


\textbf{Demostración del teorema de May de Schwartz} \\

Supón que $f$ cumple las propiedades (A), (N) y (E).

\begin{description}
\item[(a)] Si $(x_1,...,x_n)$ tiene tantos 1s como $-1$s \\
         y $(-x_1,...,-x_n)$ cambia el $1^{er}$ 1 por el $1^{er}$ $-1$, el $2^{do}$ 1 por el $2^{do}$ $-1$, etc.\\
         entonces por (A) no cambia el resultado:\\

\begin{tabular}{rcll}
$f(x_1,...,x_n)$  & = &   $f(-x_1,...,-x_n)$ & por (A) \\
  & = &   $-f(x_1,...,x_n)$   & por (N) \\
\end{tabular}

Y dado que 0 es su propio negativo:

$f(x_1,...,x_n) = 0$ 


\item[(b)] Si $(x_1,...,x_n)$ tiene $k$ más 1s que $-1$s \\
  y $(x'_1,...,x'_n)$ remplaza los primeros $k$ 1s por 0s \\
  entonces $(x'_1,...,x'_n)$ tiene tantos 1s como $-1$s \\
  y por (a) $f(x'_1,...,x'_n) = 0$ \\
  por (E) $f(x_1,...,x_n) = 1$

\item[(c)] Si $(x_1,...,x_n)$ tiene menos 1s que $-1$s \\
   entonces $(-x_1,...,-x_n)$ tiene más 1s que $-1$s \\
   y por (b) $f(-x_1,...,-x_n) = 1$ \\
   por (N) $-f(x_1,...,x_n) = 1$ \\
   y plt $f(x_1,...,x_n) = -1$ \\
\end{description}

(a) + (b) + (c) define la regla de mayoría QED

\end{document}


